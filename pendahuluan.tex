\begin{frame}
    \frametitle{Rumusan Masalah}
    \begin{enumerate}
        \item mobil otonom memerlukan sensor yang mampu menggantikan fungsi dan kemampuan dari indera manusia, terutama kemampuan untuk memperkirakan jarak halangan di sekitarnya.
        \item lidar dapat menentukan jarak secara akurat, namun tidak dapat membedakan jenis dari halangan.
        \item kamera dengan klasifikasi objek mampu membedakan jenis halangan, namun tidak dapat menentukan jarak secara akurat.
        \item dibutuhkan metode tracking untuk melakukan track pada halangan yang telah terdeteksi terhadap referensi bumi agar mobil dapat menghindari halangan.
    \end{enumerate}
\end{frame}


\begin{frame}
    \frametitle{Tujuan}
    \justifying
    Tujuan dari penelitian ini adalah \textbf{melakukan tracking halangan, posisi, dan pose dari mobil otonom} yang didapatkan melalui gabungan sensor Kamera, Lidar, GPS serta INS \textbf{menggunakan Multi-Object Integrated Probability Data Association} sehingga hasil tracking menjadi lebih akurat dibandingkan dengan data yang dihasilkan oleh satu sensor.
\end{frame}


\begin{frame}
    \frametitle{Batasan Masalah}
    \justifying
    \begin{enumerate}
        \justifying
        \item Sensor deteksi jarak menggunakan Lidar dan Kamera.
        \item Rentang deteksi Lidar dibatasi pada kemampuan sensor yang dimiliki.
        \item Rentang deteksi Kamera dibatasi pada \textit{Field of View} dan resolusi kamera.
        \item Objek tracking terklasifikasi dibatasi pada mobil dan pedestrian, namun tidak menutup kemungkinan untuk dilakukan penambahan klasifikasi lainnya.
        \item Pengambilan data akan dilakukan secara nyata, namun tidak menutup kemungkinan akan digunakan simulasi menggunakan Microsoft AirSim apabila kemampuan deteksi sensor yang dibutuhkan (Lidar, Kamera, INS dan GPS) tidak mencukupi.
    \end{enumerate}
\end{frame}


\begin{frame}
    \frametitle{Kontribusi}

    \justifying
    Kontribusi dari penelitian ini adalah menghasilkan sistem tracking halangan, posisi, dan pose kendaraan yang didapatkan melalui gabungan sensor Kamera, Lidar, GPS, serta INS menggunakan Multi-Object Integrated Probability Data Association sehingga hasil tracking menjadi lebih akurat dibandingkan dengan data yang dihasilkan oleh satu sensor.

\end{frame}