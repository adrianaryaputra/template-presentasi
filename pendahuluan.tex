\begin{frame}
    \frametitle{Rumusan Masalah}
    \begin{enumerate}
        \item Mobil otonom memerlukan sensor yang mampu menggantikan fungsi dan kemampuan dari indera manusia, terutama kemampuan untuk memperkirakan jarak halangan di sekitarnya.
        \item Lidar dapat menentukan jarak secara akurat, namun tidak dapat membedakan jenis dari halangan.
        \item Kamera dengan klasifikasi objek mampu membedakan jenis halangan, namun tidak dapat menentukan jarak secara akurat.
        \item Dibutuhkan metode untuk melakukan track pada halangan yang telah terdeteksi agar mobil dapat mengetahui halangan di sekitarnya.
        \item Waktu dan periode pengukuran dari setiap sensor berbeda-beda.
    \end{enumerate}
\end{frame}


\begin{frame}
    \frametitle{Tujuan}
    \justifying
    Tujuan dari penelitian ini adalah melakukan tracking halangan, posisi, dan pose kendaraan yang didapatkan melalui gabungan sensor Kamera, Lidar, GPS serta INS menggunakan Extended Kalman Filter Multi-Object Tracking pada data eksternal, dan Extended Kalman Filter (EKF) pada data internal dengan mempertimbangkan perbedaan waktu dan periode pengukuran tiap sensor sehingga hasil tracking menjadi lebih akurat dibandingkan dengan data yang dihasilkan oleh satu sensor tanpa mempertimbangkan perbedaan waktu pengambilan sampel deteksi.
\end{frame}


\begin{frame}
    \frametitle{Batasan Masalah}
    \justifying
    \begin{enumerate}
        \justifying
        \item Sensor deteksi jarak menggunakan Lidar.
        \item Rentang deteksi Lidar dibatasi pada kemampuan sensor yang dimiliki.
        \item Objek tracking terklasifikasi dibatasi pada mobil dan pedestrian, namun tidak menutup kemungkinan untuk dilakukan penambahan klasifikasi lainnya.
        \item Pengambilan data akan dilakukan secara nyata, namun tidak menutup kemungkinan akan digunakan simulasi menggunakan Microsoft AirSim apabila kemampuan deteksi sensor yang dibutuhkan tidak mencukupi.
    \end{enumerate}
\end{frame}


\begin{frame}
    \frametitle{Kontribusi}

    \justifying
    Kontribusi dari penelitian ini menghasilkan sistem tracking halangan, posisi, dan pose kendaraan yang didapatkan melalui gabungan sensor Kamera, Lidar, GPS, serta INS menggunakan Kalman filter dengan mempertimbangkan perbedaan waktu sampel setiap sensor sehingga hasil tracking menjadi lebih akurat dibandingkan dengan tanpa mempertimbangkan perbedaan waktu pengambilan sampel deteksi.


\end{frame}